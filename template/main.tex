% ===================================================================
% TEMPLATE ABNT PARA IGCE-UNESP
% Desenvolvido para uso no Overleaf
% Última atualização: 2025
% Autor: Gabriel Passos de Oliveira
% Email: gabriel.passos@unesp.br
% GitHub: https://github.com/gpassos01
%
% Este template está totalmente comentado para facilitar o uso por
% pessoas que nunca trabalharam com LaTeX antes.
% ===================================================================

% Comando necessário para corrigir problemas de encoding em alguns casos
\UseRawInputEncoding

% Classe do documento - abntex2 é a classe padrão para documentos ABNT no Brasil
% [12pt] = tamanho da fonte (12 pontos conforme ABNT)
% [a4paper] = papel A4 (padrão brasileiro)
% [brazil] = idioma português brasileiro
% [oneside] = impressão em apenas um lado da folha
\documentclass[12pt, a4paper, brazil, oneside]{abntex2}

% ===================================================================
% PACOTES ESSENCIAIS
% ===================================================================

% Pacotes para codificação de caracteres e idioma
\usepackage[utf8]{inputenc}        % Permite usar acentos diretamente no texto
\usepackage[T1]{fontenc}           % Codificação de fonte para caracteres especiais
\usepackage[brazil]{babel}         % Hifenização e termos em português brasileiro

% ===================================================================
% OPÇÕES DE FONTES (ESCOLHA UMA - TODAS SÃO ACEITAS PELA ABNT)
% ===================================================================

% OPÇÃO 1: Times New Roman (padrão acadêmico mais comum)
\usepackage{times}                 % Fonte Times New Roman (similar)

% OPÇÃO 2: Arial/Helvetica (descomente para usar)
% \usepackage{helvet}              % Fonte Helvetica (similar ao Arial)
% \renewcommand{\familydefault}{\sfdefault}

% OPÇÃO 3: Computer Modern (padrão LaTeX - também aceita)
% (não precisa de pacote, é a fonte padrão)

% OPÇÃO 4: Latin Modern (versão melhorada da Computer Modern)
% \usepackage{lmodern}

% ===================================================================
% PACOTES PARA FORMATAÇÃO E APARÊNCIA
% ===================================================================

\usepackage{graphicx}              % Para inserir imagens e gráficos
\usepackage{geometry}              % Para controlar margens e layout da página
\usepackage{hyperref}              % Para criar links clicáveis e sumário interativo

% ===================================================================
% SISTEMA DE CITAÇÕES (ESCOLHA UM ESTILO)
% ===================================================================

% OPÇÃO 1: Sistema autor-data (mais comum em humanas/sociais)
\usepackage[alf]{abntex2cite}      % Sistema ABNT autor-data: (SILVA, 2020)

% OPÇÃO 2: Sistema numérico (mais comum em exatas/engenharias)
% \usepackage[num]{abntex2cite}    % Sistema ABNT numérico: [1], [2], etc.

% ===================================================================
% PACOTES BÁSICOS (SEMPRE NECESSÁRIOS)
% ===================================================================

\usepackage{indentfirst}           % Indenta o primeiro parágrafo de cada seção
\usepackage{microtype}             % Melhora a qualidade da tipografia
\usepackage{xcolor}                % Para definir cores personalizadas

% ===================================================================
% PACOTES PARA MATEMÁTICA E CIÊNCIAS EXATAS
% ===================================================================

\usepackage{amsmath}               % Ambientes e comandos matemáticos avançados
\usepackage{amssymb}               % Símbolos matemáticos adicionais
\usepackage{amsfonts}              % Fontes matemáticas adicionais

% Descomente os pacotes abaixo conforme sua área:

% Para áreas com muita matemática (Física, Matemática, Engenharias):
% \usepackage{mathtools}           % Ferramentas matemáticas avançadas
% \usepackage{physics}             % Notação física (derivadas, vetores, etc.)
% \usepackage{siunitx}             % Sistema Internacional de Unidades
% \usepackage{cancel}              % Para "cancelar" termos em equações

% Para Química:
% \usepackage[version=4]{mhchem}   % Fórmulas químicas: \ce{H2SO4}
% \usepackage{chemfig}             % Estruturas químicas

% Para Estatística:
% \usepackage{pgfplots}            % Gráficos estatísticos avançados
% \usepackage{booktabs}            % Tabelas profissionais

% ===================================================================
% PACOTES PARA PROGRAMAÇÃO E COMPUTAÇÃO
% ===================================================================

\usepackage{listings}              % Para inserir códigos com formatação

% Para áreas de Computação/Tecnologia (descomente se necessário):
% \usepackage{algorithm}           % Algoritmos em pseudocódigo
% \usepackage{algorithmic}         % Comandos para algoritmos
% \usepackage{url}                 % URLs com quebra de linha
% \usepackage{verbatim}            % Texto literal (código inline)

% ===================================================================
% PACOTES PARA CIÊNCIAS HUMANAS E SOCIAIS
% ===================================================================

% Para Linguística:
% \usepackage{gb4e}                % Exemplos linguísticos numerados
% \usepackage{tipa}                % Símbolos fonéticos IPA

% Para citações especiais:
% \usepackage{csquotes}            % Citações contextuais
% \usepackage{epigraph}            % Epígrafes em capítulos

% Pacotes para controle de espaçamento e posicionamento
\usepackage{caption}               % Para personalizar legendas de figuras e tabelas
\usepackage{setspace}              % Para controlar espaçamento entre linhas
\usepackage{float}                 % Para controlar posicionamento de figuras
\usepackage{placeins}              % Para controlar quebras de página
\usepackage{underscore}            % Para usar underscore sem problemas
\usepackage{enumitem}              % Para personalizar listas numeradas e não numeradas

% ===================================================================
% COMANDOS ÚTEIS PARA ESCRITA CIENTÍFICA
% ===================================================================

% Diferencial não-itálico (boa prática matemática)
\newcommand{\dd}{\mathrm{d}}

% Comando para citações com espaço não-quebrável automático
\newcommand{\citep}[1]{~\cite{#1}}

% Comandos para referências com espaço automático
\newcommand{\figref}[1]{Fig.~\ref{#1}}
\newcommand{\tabref}[1]{Tab.~\ref{#1}}
\newcommand{\eqnref}[1]{Eq.~\eqref{#1}}
\newcommand{\secref}[1]{Seção~\ref{#1}}

% Comando para unidades com espaço não-quebrável
\newcommand{\unit}[1]{\,\mathrm{#1}}

% Exemplo de uso:
% Integral: \int f(x) \, \dd x
% Referência: \figref{fig:resultado}
% Unidade: 25\unit{°C}

% ===================================================================
% CONFIGURAÇÕES PARA CÓDIGOS DE PROGRAMAÇÃO
% ===================================================================

% Definição de cores personalizadas para destacar códigos
\definecolor{codegreen}{rgb}{0,0.6,0}      % Verde para comentários
\definecolor{codegray}{rgb}{0.5,0.5,0.5}   % Cinza para números de linha
\definecolor{codepurple}{rgb}{0.58,0,0.82} % Roxo para strings
\definecolor{backcolour}{rgb}{0.95,0.95,0.92} % Cor de fundo (bege claro)

% ===================================================================
% CONFIGURAÇÕES PARA CÓDIGOS - ESCOLHA SEU ESTILO
% ===================================================================

% ESTILO 1: Com numeração de linhas (recomendado para trabalhos técnicos)
\lstset{
  backgroundcolor=\color{backcolour},    % Cor de fundo dos códigos
  basicstyle=\footnotesize\ttfamily,     % Fonte pequena e monoespaçada
  lineskip=1pt,                          % Espaçamento entre linhas
  breakatwhitespace=false,               % Não quebra linha apenas em espaços
  breaklines=true,                       % Quebra linhas longas automaticamente
  captionpos=t,                          % Posição da legenda (top = topo)
  commentstyle=\color{codegreen}\itshape, % Estilo dos comentários (verde e itálico)
  deletekeywords={...},                  % Remove palavras-chave específicas
  escapeinside={\%*}{*)-},               % Permite LaTeX dentro do código
  extendedchars=true,                    % Suporte a caracteres especiais
  frame=single,                          % Borda simples ao redor do código
  keepspaces=true,                       % Preserva espaços no código
  keywordstyle=\color{blue},             % Palavras-chave em azul
  language=Python,                       % ALTERE: Python, Java, C, C++, JavaScript, etc.
  numbers=none,                          % ALTERE: none, left, right
  % Tratamento de caracteres especiais e acentos
  literate={_}{\_}1 {á}{{\'a}}1 {é}{{\'e}}1 {í}{{\'i}}1 {ó}{{\'o}}1 {ú}{{\'u}}1 {ã}{{\~a}}1 {õ}{{\~o}}1 {ç}{{\c{c}}}1,
  numbersep=5pt,                         % Distância entre números e código
  numberstyle=\tiny\color{codegray},     % Estilo dos números de linha
  rulecolor=\color{black},               % Cor da borda
  showspaces=false,                      % Não mostra espaços
  showstringspaces=false,                % Não mostra espaços em strings
  showtabs=false,                        % Não mostra tabs
  stepnumber=1,                          % Intervalo de numeração (1=todas, 5=a cada 5, etc.)
  stringstyle=\color{codepurple},        % Cor das strings
  tabsize=2,                             % Tamanho do tab (2 ou 4 são comuns)
  title=\lstname,                        % Título baseado no nome do arquivo
  frameround=tttt,                       % Cantos arredondados da borda
  % Margens internas da caixa de código
  framexleftmargin=8pt,
  framexrightmargin=8pt,
  framextopmargin=8pt,
  framexbottommargin=8pt,
  xleftmargin=8pt,                       % Margem esquerda
  xrightmargin=8pt,                      % Margem direita
  aboveskip=12pt,                        % Espaço antes do código
  belowskip=12pt,                        % Espaço depois do código
  float                                  % Permite que o código "flutue" na página
}

% ===================================================================
% CONFIGURAÇÕES ALTERNATIVAS PARA CÓDIGOS (DESCOMENTE PARA USAR)
% ===================================================================

% ESTILO 2: Sem borda (mais limpo)
% \lstset{
%   backgroundcolor=\color{backcolour},
%   basicstyle=\footnotesize\ttfamily,
%   breaklines=true,
%   commentstyle=\color{codegreen}\itshape,
%   keywordstyle=\color{blue},
%   language=Python,
%   numbers=left,
%   numberstyle=\tiny\color{codegray},
%   stringstyle=\color{codepurple},
%   frame=none,                          % Sem borda
%   xleftmargin=0pt,
%   framexleftmargin=0pt
% }

% ESTILO 3: Monocromático (para impressão P&B)
% \lstset{
%   backgroundcolor=\color{white},
%   basicstyle=\footnotesize\ttfamily,
%   breaklines=true,
%   commentstyle=\itshape,               % Sem cor, apenas itálico
%   keywordstyle=\bfseries,              % Sem cor, apenas negrito
%   language=Python,
%   numbers=left,
%   numberstyle=\tiny,
%   stringstyle=\itshape,
%   frame=single
% }

% ===================================================================
% CONFIGURAÇÕES DE LAYOUT DA PÁGINA (MARGENS ABNT)
% ===================================================================

% Definição das margens conforme normas ABNT
\geometry{
    a4paper,          % Tamanho do papel A4
    left=3cm,         % Margem esquerda: 3cm (padrão ABNT)
    right=2cm,        % Margem direita: 2cm (padrão ABNT)
    top=3cm,          % Margem superior: 3cm (padrão ABNT)
    bottom=2cm,       % Margem inferior: 2cm (padrão ABNT)
    headheight=15pt   % Altura do cabeçalho
}

% ===================================================================
% CONFIGURAÇÕES DE ESPAÇAMENTO E PARÁGRAFOS
% ===================================================================

% Recuo da primeira linha de cada parágrafo (1,25cm conforme ABNT)
\setlength{\parindent}{1.25cm}

% Espaçamento entre parágrafos (ABNT recomenda 0pt - sem espaço extra)
\setlength{\parskip}{0pt}

% ===================================================================
% CONFIGURAÇÕES DE LISTAS (ITEMIZE E ENUMERATE)
% ===================================================================

% Configurações para listas não numeradas (itemize)
% leftmargin: alinhamento da margem esquerda
% itemsep: espaço entre itens
% parsep: espaço entre parágrafos dentro de um item
% topsep: espaço antes e depois da lista
\setlist[itemize]{leftmargin=1.25cm, itemsep=0pt, parsep=0pt, topsep=0pt}

% Configurações para listas numeradas (enumerate) - mesmas configurações
\setlist[enumerate]{leftmargin=1.25cm, itemsep=0pt, parsep=0pt, topsep=0pt}

% ===================================================================
% CONFIGURAÇÕES DE FONTES DOS TÍTULOS (ABNT)
% ===================================================================

% Fonte dos títulos dos capítulos: romana (Times), negrito
\renewcommand{\ABNTEXchapterfont}{\rmfamily\bfseries}
% Tamanho da fonte dos capítulos: large (maior que o texto normal)
\renewcommand{\ABNTEXchapterfontsize}{\large}

% Fonte dos títulos das seções: romana (Times), negrito
\renewcommand{\ABNTEXsectionfont}{\rmfamily\bfseries}
% Tamanho da fonte das seções: normal (mesmo tamanho do texto)
\renewcommand{\ABNTEXsectionfontsize}{\normalsize}

% Fonte dos títulos das subseções: romana (Times), negrito
\renewcommand{\ABNTEXsubsectionfont}{\rmfamily\bfseries}
% Tamanho da fonte das subseções: normal
\renewcommand{\ABNTEXsubsectionfontsize}{\normalsize}

% Fonte das subsubseções: romana (Times), SEM negrito
\renewcommand{\ABNTEXsubsubsectionfont}{\rmfamily}
% Tamanho da fonte das subsubseções: normal
\renewcommand{\ABNTEXsubsubsectionfontsize}{\normalsize}

% ===================================================================
% CONFIGURAÇÕES DE HYPERLINKS E PROPRIEDADES DO PDF
% ===================================================================

\hypersetup{
    colorlinks=true,       % Ativa links coloridos (em vez de caixas coloridas)
    citecolor=black,       % Cor das citações: preto
    linkcolor=blue,        % Cor dos links internos: azul
    filecolor=magenta,     % Cor dos links para arquivos: magenta
    urlcolor=cyan,         % Cor dos links para URLs: ciano
    % Metadados do PDF (ALTERE CONFORME SEU TRABALHO)
    pdftitle={TÍTULO DO SEU TRABALHO AQUI},
    pdfauthor={SEU NOME AQUI},
    pdfsubject={Trabalho Acadêmico - IGCE UNESP},
    pdfkeywords={palavra-chave1, palavra-chave2, palavra-chave3}
}

% ===================================================================
% CONFIGURAÇÕES DA CAPA - ADAPTÁVEL PARA DIFERENTES INSTITUIÇÕES
% ===================================================================

% ===================================================================
% OPÇÕES DE CONFIGURAÇÃO DA CAPA (PERSONALIZE CONFORME SUA INSTITUIÇÃO)
% ===================================================================

% Para IGCE-UNESP (padrão atual):
\newcommand{\logoesquerda}{unesp_logo.jpg}        % Arquivo do logo esquerdo
\newcommand{\logodireita}{igce_brasao.jpg}        % Arquivo do logo direito (ou deixe vazio)
\renewcommand{\instituicao}{UNIVERSIDADE ESTADUAL PAULISTA "JÚLIO DE MESQUITA FILHO"}
\newcommand{\faculdade}{INSTITUTO DE GEOCIÊNCIAS E CIÊNCIAS EXATAS}
\newcommand{\cidade}{Rio Claro - SP}

% Para outras instituições (descomente e adapte):

% Para outras unidades da UNESP:
% \renewcommand{\logoesquerda}{unesp_logo.jpg}
% \renewcommand{\logodireita}{logo_sua_unidade.jpg}
% \renewcommand{\faculdade}{NOME DA SUA FACULDADE/INSTITUTO}
% \renewcommand{\cidade}{Sua Cidade - SP}

% Para USP:
% \renewcommand{\logoesquerda}{usp_logo.jpg}
% \renewcommand{\logodireita}{logo_sua_unidade.jpg}
% \renewcommand{\instituicao}{UNIVERSIDADE DE SÃO PAULO}
% \renewcommand{\faculdade}{NOME DA SUA FACULDADE/INSTITUTO}
% \renewcommand{\cidade}{Sua Cidade - SP}

% Para UNICAMP:
% \renewcommand{\logoesquerda}{unicamp_logo.jpg}
% \renewcommand{\logodireita}{}
% \renewcommand{\instituicao}{UNIVERSIDADE ESTADUAL DE CAMPINAS}
% \renewcommand{\faculdade}{NOME DA SUA FACULDADE/INSTITUTO}
% \renewcommand{\cidade}{Campinas - SP}

% Para universidades privadas ou outras públicas:
% \renewcommand{\logoesquerda}{logo_sua_universidade.jpg}
% \renewcommand{\logodireita}{}
% \renewcommand{\instituicao}{NOME DA SUA UNIVERSIDADE}
% \renewcommand{\faculdade}{NOME DA SUA FACULDADE/CURSO}
% \renewcommand{\cidade}{Sua Cidade - Estado}

% ===================================================================
% IMPLEMENTAÇÃO DA CAPA
% ===================================================================

\renewcommand{\imprimircapa}{
    \newpage                    % Inicia uma nova página
    \thispagestyle{empty}       % Remove numeração da página
    \begin{center}              % Centraliza todo o conteúdo
        \vspace*{-2cm}          % Ajuste fino da posição vertical
        
        % ===== CABEÇALHO COM LOGOS =====
        \ifx\logodireita\empty
            % Apenas um logo centralizado
            \includegraphics[height=3cm]{\logoesquerda}
        \else
            % Dois logos lado a lado
            \begin{minipage}[c]{0.45\textwidth}
                \centering
                \includegraphics[height=3cm]{\logoesquerda} 
            \end{minipage}
            \hspace{0.05\textwidth}  % Espaçamento entre os logos
            \begin{minipage}[c]{0.45\textwidth}
                \centering
                \includegraphics[height=3cm]{\logodireita}
            \end{minipage}
        \fi
        
        \vspace{4cm}            % Espaço entre cabeçalho e autores
        
        % ===== NOME(S) DO(S) AUTOR(ES) =====
        % ALTERE AQUI COM SEU(S) NOME(S)
        {\large NOME DO PRIMEIRO AUTOR}
        \vspace{0.2cm}\par      % Pequeno espaço entre autores
        % Para trabalhos em grupo, descomente as linhas abaixo:
        % {\large NOME DO SEGUNDO AUTOR}
        % \vspace{0.2cm}\par
        % {\large NOME DO TERCEIRO AUTOR}
        % \vspace{0.2cm}\par
        % {\large NOME DO QUARTO AUTOR}  % Adicione quantos autores necessário
        
        \vfill                  % Espaço flexível
        
        % ===== TÍTULO DO TRABALHO =====
        % ALTERE AQUI COM O TÍTULO DO SEU TRABALHO
        {\large \bfseries TÍTULO DO SEU TRABALHO ACADÊMICO\par}
        
        \vfill                  % Espaço flexível
        \vfill                  % Mais espaço flexível (empurra para baixo)
        
        % ===== CIDADE E ANO =====
        {\large \cidade}\\
        {\large 2025}          % ALTERE O ANO CONFORME NECESSÁRIO
    \end{center}
    \newpage                   % Vai para a próxima página
}

% ===================================================================
% CONFIGURAÇÕES DA FOLHA DE ROSTO - ADAPTÁVEL PARA DIFERENTES TRABALHOS
% ===================================================================

% ===================================================================
% CONFIGURAÇÕES PARA DIFERENTES TIPOS DE TRABALHO
% ===================================================================

% Defina aqui o tipo do seu trabalho (descomente apenas uma opção):

% OPÇÃO 1: Trabalho de Conclusão de Curso (TCC)
\renewcommand{\tipotrabalho}{Trabalho de Conclusão de Curso}
\newcommand{\grautrabalho}{Bacharelado}
\newcommand{\cursotrabalho}{NOME DO SEU CURSO}
\newcommand{\orientacao}{Orientador: NOME DO PROFESSOR ORIENTADOR}

% OPÇÃO 2: Dissertação de Mestrado (descomente se for o caso)
% \renewcommand{\tipotrabalho}{Dissertação}
% \renewcommand{\grautrabalho}{Mestrado}
% \renewcommand{\cursotrabalho}{NOME DO PROGRAMA DE PÓS-GRADUAÇÃO}
% \renewcommand{\orientacao}{Orientador: NOME DO PROFESSOR ORIENTADOR}

% OPÇÃO 3: Tese de Doutorado (descomente se for o caso)
% \renewcommand{\tipotrabalho}{Tese}
% \renewcommand{\grautrabalho}{Doutorado}
% \renewcommand{\cursotrabalho}{NOME DO PROGRAMA DE PÓS-GRADUAÇÃO}
% \renewcommand{\orientacao}{Orientador: NOME DO PROFESSOR ORIENTADOR}

% OPÇÃO 4: Relatório de Iniciação Científica
% \renewcommand{\tipotrabalho}{Relatório de Iniciação Científica}
% \renewcommand{\grautrabalho}{}
% \renewcommand{\cursotrabalho}{NOME DO SEU CURSO}
% \renewcommand{\orientacao}{Orientador: NOME DO PROFESSOR ORIENTADOR}

% OPÇÃO 5: Trabalho para Disciplina
% \renewcommand{\tipotrabalho}{Trabalho}
% \renewcommand{\grautrabalho}{}
% \renewcommand{\cursotrabalho}{NOME DO SEU CURSO}
% \renewcommand{\orientacao}{Docente: NOME DO PROFESSOR}

% ===================================================================
% IMPLEMENTAÇÃO DA FOLHA DE ROSTO
% ===================================================================

\renewcommand{\imprimirfolhaderosto}{
    \newpage                    % Inicia uma nova página
    \thispagestyle{empty}       % Remove numeração da página
    \begin{center}              % Centraliza todo o conteúdo
        \vspace*{-2cm}          % Ajuste fino da posição vertical
        
        % ===== CABEÇALHO COM LOGOS (MESMO DA CAPA) =====
        \ifx\logodireita\empty
            % Apenas um logo centralizado
            \includegraphics[height=3cm]{\logoesquerda}
        \else
            % Dois logos lado a lado
            \begin{minipage}[c]{0.45\textwidth}
                \centering
                \includegraphics[height=3cm]{\logoesquerda} 
            \end{minipage}
            \hspace{0.05\textwidth}  % Espaçamento entre os logos
            \begin{minipage}[c]{0.45\textwidth}
                \centering
                \includegraphics[height=3cm]{\logodireita}
            \end{minipage}
        \fi
        
        \vspace{2cm}            % Espaço menor que na capa
        
        % ===== NOME(S) DO(S) AUTOR(ES) =====
        % ALTERE AQUI COM SEU(S) NOME(S) (MESMOS DA CAPA)
        {\large NOME DO PRIMEIRO AUTOR}
        \vspace{0.2cm}\par
        % Para trabalhos em grupo, descomente as linhas abaixo:
        % {\large NOME DO SEGUNDO AUTOR}
        % \vspace{0.2cm}\par
        % {\large NOME DO TERCEIRO AUTOR}
        % \vspace{0.2cm}\par
        % {\large NOME DO QUARTO AUTOR}
        
        \vspace{1.5cm}          % Espaço entre autores e título
        
        % ===== TÍTULO DO TRABALHO =====
        % ALTERE AQUI COM O TÍTULO DO SEU TRABALHO (MESMO DA CAPA)
        {\large \bfseries TÍTULO DO SEU TRABALHO ACADÊMICO\par}
        
        \vspace{1.5cm}          % Espaço entre título e natureza do trabalho
        
        % ===== NATUREZA DO TRABALHO =====
        % Esta seção é gerada automaticamente baseada no tipo de trabalho definido acima
        \begin{flushright}      % Alinha à direita
            \begin{minipage}{0.5\textwidth}  % Ocupa metade da largura
                \linespread{1.3}\selectfont   % Espaçamento entre linhas 1.3
                \tipotrabalho\space apresentado ao Curso de \cursotrabalho\space da \faculdade, \instituicao, como requisito parcial para obtenção do título de \ifx\grautrabalho\empty\else\grautrabalho\space em \cursotrabalho\fi.
                \vspace{1cm}\par  % Espaço antes do nome do docente
                \orientacao
            \end{minipage}
        \end{flushright}

        \vfill                  % Espaço flexível (empurra cidade/ano para baixo)
        
        % ===== CIDADE E ANO =====
        {\large \cidade}\\
        {\large 2025}          % ALTERE O ANO CONFORME NECESSÁRIO
    \end{center}
    \newpage                   % Vai para a próxima página
}

% ===================================================================
% CONFIGURAÇÕES DA BIBLIOGRAFIA (SISTEMA ABNT)
% ===================================================================

% Define o estilo da bibliografia conforme ABNT (autor-data)
\bibliographystyle{abntex2-alf}

% ===================================================================
% INÍCIO DO DOCUMENTO
% ===================================================================

\begin{document}

% Define espaçamento 1.5 entre linhas (conforme exigência ABNT)
\OnehalfSpacing

% ===================================================================
% ELEMENTOS PRÉ-TEXTUAIS
% ===================================================================

% Imprime a capa (primeira página)
\imprimircapa

% Imprime a folha de rosto (segunda página)
\imprimirfolhaderosto

% ===================================================================
% RESUMO
% ===================================================================

\begin{resumo}
% ESCREVA AQUI O RESUMO DO SEU TRABALHO
% O resumo deve ter entre 150 a 500 palavras
% Deve apresentar objetivos, metodologia, resultados e conclusões
% Seja objetivo e claro

Escreva aqui o resumo do seu trabalho. O resumo deve apresentar de forma concisa os objetivos, a metodologia utilizada, os principais resultados obtidos e as conclusões do trabalho. Deve ter entre 150 a 500 palavras e ser escrito em parágrafo único, sem citações ou referências.

\vspace{\onelineskip}    % Espaço de uma linha antes das palavras-chave
\noindent               % Remove indentação da linha
% ALTERE AS PALAVRAS-CHAVE CONFORME SEU TRABALHO (3 a 5 palavras)
\textbf{Palavras-chave:} palavra-chave1, palavra-chave2, palavra-chave3, palavra-chave4.

\end{resumo}

% ===================================================================
% LISTAS DE FIGURAS E TABELAS (OPCIONAL)
% ===================================================================

\newpage
% Descomente a linha abaixo se seu trabalho tiver figuras
% \listoffigures

\newpage
% Descomente a linha abaixo se seu trabalho tiver tabelas
% \listoftables

% ===================================================================
% LISTA DE ABREVIATURAS E SIGLAS (OPCIONAL)
% ===================================================================

\newpage
% Esta seção é opcional - use apenas se seu trabalho tiver muitas siglas
\chapter*{LISTA DE ABREVIATURAS E SIGLAS}
% Adicione suas siglas em ordem alfabética
% Formato: SIGLA & Significado completo\\
\begin{tabular}{ll}
ABNT    & Associação Brasileira de Normas Técnicas\\
IGCE    & Instituto de Geociências e Ciências Exatas\\
UNESP   & Universidade Estadual Paulista\\
% Adicione mais siglas conforme necessário
\end{tabular}

% ===================================================================
% SUMÁRIO
% ===================================================================

\newpage
% Redefine o título do sumário para português
\renewcommand{\contentsname}{SUMÁRIO}
% Gera o sumário automaticamente
\tableofcontents

% ===================================================================
% INÍCIO DO TEXTO PRINCIPAL
% ===================================================================

\newpage
% Define o estilo de cabeçalho das páginas (numeração no topo)
\pagestyle{abntheadings}
% Marca o início da parte textual do documento
\textual

% ===================================================================
% BOAS PRÁTICAS PARA ESCRITA CIENTÍFICA EM LaTeX
% ===================================================================
% 
% 📝 TEXTO E ESPAÇAMENTO:
% • Use ~ para evitar quebras indesejadas: "Fig.~\ref{fig:exemplo}"
% • Para abreviações: "e.g.\" (barra evita espaço extra)
% • Aspas corretas: ``texto'' para aspas duplas
% • Parágrafos: linha em branco no código (não \newline)
% 
% 🔢 EQUAÇÕES:
% • Sempre adicione pontuação: \begin{equation} E = mc^2 \,. \end{equation}
% • Diferencial não-itálico: \int f(x) \, \mathrm{d}x
% • Espaço antes da pontuação: \,. ou \,,
% • Subscripts descritivos: N_\mathrm{partículas}
% 
% 📊 REFERÊNCIAS:
% • Labels descritivos: \label{eq:teorema_principal}
% • Use \eqref{} para equações: Eq.~\eqref{eq:exemplo}
% • Sempre com ~: "veja Fig.~\ref{fig:dados}"
% • Início de frase: "A Equação 1 mostra..." (por extenso)
% 
% 🖼️ FIGURAS E TABELAS:
% • Placement [tb] (topo/base) melhor que [h]
% • Sempre com \label{fig:nome_descritivo}
% • Caption informativo antes de tabelas
% • Fonte obrigatória: \caption*{Fonte: ...}
% 
% 📝 CITAÇÕES:
% • Sempre com ~: Silva~\cite{silva2023}
% • Múltiplas: \cite{ref1,ref2,ref3}
% • No início: "Silva (2023) propôs..."
% 
% ===================================================================

% ===================================================================
% EXEMPLOS DE ELEMENTOS COMUNS EM TRABALHOS ACADÊMICOS
% ===================================================================

% EXEMPLO DE COMO INSERIR FIGURA (boas práticas)
% Descomente e adapte conforme necessário
% \begin{figure}[tb]                                    % [tb] = topo ou base (melhor que [h])
%   \centering                                          % Centraliza a figura
%   \includegraphics[width=0.7\textwidth]{exemplo.png} % Insere imagem
%   \caption{Descrição clara e informativa da figura}  % Legenda da figura
%   \caption*{Fonte: Elaboração própria.}              % Fonte (obrigatório ABNT)
%   \label{fig:exemplo}                                 % Label para referência
% \end{figure}
%
% No texto, sempre referencie: "A Fig.~\ref{fig:exemplo} mostra que..."
% Para duas colunas, use figure* para figura em página inteira

% ===================================================================
% CAPÍTULOS DO SEU TRABALHO
% ===================================================================

\chapter{INTRODUÇÃO}

% ESCREVA AQUI A INTRODUÇÃO DO SEU TRABALHO
% A introdução deve:
% - Contextualizar o tema
% - Apresentar a relevância do assunto
% - Introduzir o problema de pesquisa
% - Apresentar objetivos
% - Descrever a organização do trabalho

Escreva aqui a introdução do seu trabalho. A introdução deve contextualizar o tema abordado, apresentar a relevância e importância do assunto, e introduzir gradualmente o leitor ao problema de pesquisa.
%
Este parágrafo deve apresentar o contexto geral do tema, situando o leitor na área de conhecimento. É importante demonstrar por que o tema é relevante e atual. Use aspas corretamente: ``exemplo de citação'' ou \enquote{exemplo} se usar csquotes.
%
O próximo parágrafo pode abordar trabalhos relacionados ou o estado da arte na área, identificando lacunas ou oportunidades de pesquisa. Para abreviações como e.g.\ ou i.e.\, use barra invertida para espaçamento correto.

\section{Problema e Justificativa}

Nesta seção, apresente claramente o problema de pesquisa que seu trabalho pretende abordar. Explique por que este problema precisa ser investigado e qual a importância de encontrar uma solução.

Justifique a escolha do tema e demonstre como seu trabalho pode contribuir para o avanço do conhecimento na área.

\section{Objetivos}

\subsection{Objetivo Geral}

O objetivo geral deve expressar o que se pretende alcançar com o trabalho de forma ampla. Exemplo:

O objetivo principal deste trabalho é [descrever o objetivo principal de forma clara e específica].

\subsection{Objetivos Específicos}

Os objetivos específicos detalham as etapas necessárias para alcançar o objetivo geral:

\begin{itemize}
    \item Primeiro objetivo específico
    \item Segundo objetivo específico  
    \item Terceiro objetivo específico
    \item Adicione quantos objetivos específicos forem necessários
\end{itemize}

\section{Metodologia}

Apresente brevemente a metodologia que será utilizada no trabalho. Esta seção pode ser expandida em um capítulo específico se necessário.

\section{Organização do Trabalho}

Descreva como o trabalho está organizado. Exemplo:

Este trabalho está organizado da seguinte forma: o Capítulo 2 apresenta a fundamentação teórica; o Capítulo 3 descreve a metodologia utilizada; o Capítulo 4 apresenta os resultados obtidos; e o Capítulo 5 apresenta as conclusões e trabalhos futuros.

% ===================================================================

\chapter{FUNDAMENTAÇÃO TEÓRICA}

% ESCREVA AQUI A FUNDAMENTAÇÃO TEÓRICA DO SEU TRABALHO
% Este capítulo deve apresentar:
% - Conceitos básicos da área
% - Estado da arte 
% - Trabalhos relacionados
% - Base teórica necessária para compreender seu trabalho

A fundamentação teórica estabelece as bases conceituais necessárias para a compreensão do trabalho. Este capítulo deve apresentar os principais conceitos, teorias e trabalhos relacionados ao tema abordado.

\section{Conceitos Básicos}

Apresente aqui os conceitos fundamentais necessários para compreensão do seu trabalho. Defina termos técnicos e explique teorias relevantes.

% EXEMPLOS DE CITAÇÕES (sempre com espaço não-quebrável):
% Segundo Silva~\cite{silva2023fundamentos}, a teoria X é fundamental para...
% 
% Como demonstrado por vários autores~\cite{silva2023fundamentos,chen2023deep}, 
% o método proposto...
%
% A Eq.~\eqref{eq:einstein} foi proposta por Einstein em~1905.
%
% A Fig.~\ref{fig:resultados} apresenta os dados obtidos.
%
% No início de frase: "A Equação~\eqref{eq:exemplo} demonstra que..."

\section{Estado da Arte}

Apresente aqui uma revisão da literatura sobre o tema, mostrando o que já foi pesquisado e quais são as lacunas existentes.

\section{Trabalhos Relacionados}

Compare seu trabalho com outros similares, destacando as diferenças e contribuições.

% ===================================================================

\chapter{METODOLOGIA}

% ESCREVA AQUI A METODOLOGIA DO SEU TRABALHO
% Este capítulo deve descrever:
% - Tipo de pesquisa
% - Procedimentos metodológicos
% - Materiais e métodos
% - Forma de análise dos dados

Este capítulo descreve a metodologia utilizada para o desenvolvimento do trabalho, incluindo os procedimentos, técnicas e ferramentas empregadas.

\section{Tipo de Pesquisa}

Descreva o tipo de pesquisa (qualitativa, quantitativa, experimental, etc.) e justifique a escolha.

\section{Materiais e Métodos}

Descreva os materiais utilizados, equipamentos, software, metodologias específicas, etc.

% EXEMPLO DE LISTA DE MATERIAIS
\begin{itemize}
    \item Material 1
    \item Material 2  
    \item Software utilizado
    \item Equipamentos
\end{itemize}

\section{Procedimentos}

Descreva passo a passo como o trabalho foi desenvolvido.

% ===================================================================

\chapter{RESULTADOS E DISCUSSÃO}

% ESCREVA AQUI OS RESULTADOS DO SEU TRABALHO
% Este capítulo deve apresentar:
% - Resultados obtidos
% - Análise dos resultados
% - Discussão sobre os achados
% - Comparação com trabalhos relacionados

Este capítulo apresenta os resultados obtidos durante o desenvolvimento do trabalho e sua discussão.

\section{Resultados Obtidos}

Apresente aqui os principais resultados do seu trabalho.

% EXEMPLO DE TABELA (boas práticas)
\begin{table}[tb]                                      % [tb] = topo ou base
\centering
\caption{Resultados experimentais do método proposto} % Caption descritivo antes da tabela
\label{tab:resultados_experimento}                    % Label descritivo
\begin{tabular}{|c|c|c|}
\hline
\textbf{Item} & \textbf{Valor 1} & \textbf{Valor 2} \\
\hline
A & 10 & 20 \\
B & 15 & 25 \\
C & 8 & 18 \\
\hline
\end{tabular}
\caption*{Fonte: Elaboração própria.}
\end{table}
%
% No texto: "Os resultados da Tab.~\ref{tab:resultados_experimento} indicam..."
% Alternativa com booktabs (mais profissional):
% \begin{tabular}{lcc}
% \toprule
% Item & Valor 1 & Valor 2 \\
% \midrule
% A & 10 & 20 \\
% B & 15 & 25 \\
% C & 8 & 18 \\
% \bottomrule
% \end{tabular}

\section{Análise dos Resultados}

Analise e interprete os resultados obtidos.

\section{Discussão}

Discuta os resultados, compare com a literatura e apresente suas interpretações.

% ===================================================================

\chapter{CONSIDERAÇÕES FINAIS}

% ESCREVA AQUI AS CONCLUSÕES DO SEU TRABALHO
% Este capítulo deve apresentar:
% - Síntese dos principais resultados
% - Conclusões sobre os objetivos
% - Limitações do trabalho
% - Sugestões para trabalhos futuros

Este capítulo apresenta as principais conclusões do trabalho, suas limitações e sugestões para trabalhos futuros.

\section{Síntese dos Resultados}

Faça uma síntese dos principais resultados obtidos.

\section{Conclusões}

Apresente as conclusões principais do trabalho, relacionando-as com os objetivos propostos.

\section{Limitações}

Discuta as limitações encontradas durante o desenvolvimento do trabalho.

\section{Trabalhos Futuros}

Sugira possíveis extensões ou melhorias para trabalhos futuros.

% ===================================================================
% ELEMENTOS PÓS-TEXTUAIS
% ===================================================================

% Inicia a parte pós-textual do documento
\postextual

\newpage

% ===================================================================
% REFERÊNCIAS BIBLIOGRÁFICAS
% ===================================================================

% IMPORTANTE: Para usar citações e referências, você precisa:
% 1. O arquivo referencias.bib já está incluído com 22 categorias de exemplos
% 2. Substitua os exemplos pelas suas próprias referências no formato BibTeX
% 3. Citar as referências no texto usando \cite{chave}
% 4. Compilar o documento múltiplas vezes no Overleaf

% O arquivo referencias.bib inclui exemplos de:
% - Livros, artigos científicos, teses
% - Sites governamentais e empresariais
% - Documentação técnica e software
% - Vídeos educacionais e podcasts
% - Preprints e bases de dados
% - Legislação e normas técnicas
% - E muito mais!

% Para citar no texto: \cite{chave_da_referencia}
% Exemplos: \cite{silva2023fundamentos}, \cite{github2023copilot}, \cite{ibge2023censo}

\bibliography{referencias}

% ===================================================================
% INSTRUÇÕES PARA USAR ESTE TEMPLATE NO OVERLEAF
% ===================================================================

% PASSO 1: FAZER UPLOAD DOS ARQUIVOS NECESSÁRIOS
% - template.tex (este arquivo)
% - referencias.bib (arquivo com 22 categorias de exemplos)
% - Logos da sua instituição: unesp_logo.jpg e igce_brasao.jpg
% - Coloque todos na raiz do projeto (mesmo diretório)

% PASSO 2: PERSONALIZAR O DOCUMENTO
% - Substitua "NOME DO PRIMEIRO AUTOR" pelo seu nome
% - Substitua "TÍTULO DO SEU TRABALHO" pelo título real
% - Substitua "NOME DO SEU CURSO" pelo seu curso
% - Substitua "NOME DA DISCIPLINA" pela disciplina
% - Substitua "NOME DO PROFESSOR" pelo nome do professor

% PASSO 3: ESCREVER O CONTEÚDO
% - Substitua os textos exemplo pelo conteúdo real
% - Adicione mais capítulos conforme necessário
% - Use os exemplos fornecidos como guia

% PASSO 4: PERSONALIZAR REFERÊNCIAS
% - Use o arquivo referencias.bib fornecido (22 categorias de exemplos)
% - Substitua os exemplos por suas próprias referências
% - Cite no texto usando \cite{chave}
% - Exemplos disponíveis: sites, documentação, vídeos, preprints, etc.

% PASSO 5: COMPILAR
% - No Overleaf, clique em "Recompile"
% - Se houver erros, verifique se todas as imagens foram carregadas
% - Se usar citações, compile algumas vezes para resolver referências

% DICAS IMPORTANTES:
% - Mantenha as configurações ABNT (não altere as seções iniciais)
% - Use sempre \chapter para capítulos principais
% - Use \section, \subsection, \subsubsection para subdivisões
% - Para figuras, use sempre \caption e \caption*{Fonte: ...}
% - Para tabelas, use sempre \caption e \caption*{Fonte: ...}
% - Numere as equações importantes
% - Cite sempre as fontes das informações

% ===================================================================
% EXEMPLOS DE ELEMENTOS PARA DIFERENTES ÁREAS DO CONHECIMENTO
% ===================================================================

% ===================================================================
% EXEMPLOS PARA CIÊNCIAS EXATAS E ENGENHARIAS
% ===================================================================

% EQUAÇÕES MATEMÁTICAS (com pontuação adequada):
% \begin{equation}
% \nabla \cdot \vec{E} = \frac{\rho}{\epsilon_0} \,,
% \label{eq:gauss}
% \end{equation}
% No texto: "conforme a Eq.~\eqref{eq:gauss}, o campo elétrico..."

% SISTEMA DE EQUAÇÕES:
% \begin{align}
% x + y &= 5 \,, \label{eq:sistema_a} \\
% 2x - y &= 1 \,. \label{eq:sistema_b}
% \end{align}

% INTEGRAL COM DIFERENCIAL CORRETO:
% A probabilidade é dada por
% \begin{equation}
% P = \int_{0}^{\infty} f(x) \, \mathrm{d}x \,.
% \label{eq:probabilidade}
% \end{equation}

% UNIDADES (se usar siunitx) - sempre com espaço não-quebrável:
% A temperatura foi de~\SI{25}{\celsius} e a pressão de~\SI{1.2e5}{\pascal}.

% FÓRMULAS QUÍMICAS (se usar mhchem):
% \ce{H2SO4 + 2NaOH -> Na2SO4 + 2H2O}

% SUBSCRIPTS DESCRITIVOS (não itálicos):
% $N_\mathrm{partículas}$ ao invés de $N_{partículas}$

% ===================================================================
% EXEMPLOS PARA COMPUTAÇÃO E TECNOLOGIA
% ===================================================================

% CÓDIGO DE PROGRAMAÇÃO (boas práticas):
% \begin{lstlisting}[language=Python, caption=Implementação do algoritmo de Fibonacci, label=lst:fibonacci]
% def fibonacci(n):
%     """Calcula o n-ésimo número de Fibonacci"""
%     if n <= 1:
%         return n
%     return fibonacci(n-1) + fibonacci(n-2)
% 
% # Exemplo de uso
% resultado = fibonacci(10)
% print(f"O 10º número de Fibonacci é: {resultado}")
% \end{lstlisting}
%
% No texto: "O Código~\ref{lst:fibonacci} apresenta a implementação..."
% Use labels descritivos: lst:nome_funcionalidade

% ALGORITMO EM PSEUDOCÓDIGO (se usar algorithm):
% \begin{algorithm}[tb]                                % [tb] = topo ou base
% \caption{Algoritmo de busca binária}                 % Caption descritivo
% \label{alg:busca_binaria}                            % Label para referência
% \begin{algorithmic}[1]                               % [1] = numera todas as linhas
% \REQUIRE $array$ ordenado, $valor$ a ser buscado
% \ENSURE índice do $valor$ ou $-1$ se não encontrado
% \STATE $inicio \leftarrow 0$
% \STATE $fim \leftarrow |array| - 1$
% \WHILE{$inicio \leq fim$}
%     \STATE $meio \leftarrow \lfloor(inicio + fim) / 2\rfloor$
%     \IF{$array[meio] = valor$}
%         \RETURN $meio$
%     \ELSIF{$array[meio] < valor$}
%         \STATE $inicio \leftarrow meio + 1$
%     \ELSE
%         \STATE $fim \leftarrow meio - 1$
%     \ENDIF
% \ENDWHILE
% \RETURN $-1$
% \end{algorithmic}
% \end{algorithm}
%
% No texto: "O Algoritmo~\ref{alg:busca_binaria} implementa a busca..."

% URL COM QUEBRA DE LINHA:
% Site oficial: \url{https://www.exemplo.com/pagina-muito-longa}

% ===================================================================
% EXEMPLOS PARA CIÊNCIAS HUMANAS E SOCIAIS
% ===================================================================

% CITAÇÃO DIRETA LONGA (mais de 3 linhas):
% \begin{citacao}
% Esta é uma citação longa que deve ser formatada em parágrafo
% separado, com recuo à esquerda e fonte menor, conforme as
% normas ABNT para citações diretas longas. O texto deve ter
% mais de três linhas para usar este formato.
% \end{citacao}

% EPÍGRAFE (se usar epigraph):
% \epigraph{A educação é a arma mais poderosa que você pode usar para mudar o mundo.}{Nelson Mandela}

% EXEMPLOS LINGUÍSTICOS (se usar gb4e):
% \begin{exe}
% \ex O menino comprou o livro.
% \ex *O menino compraram o livro.
% \end{exe}

% ===================================================================
% EXEMPLOS GERAIS PARA TODAS AS ÁREAS
% ===================================================================

% FIGURA COM REFERÊNCIA:
% \begin{figure}[h]
%   \centering
%   \includegraphics[width=0.7\textwidth]{minha_figura.png}
%   \caption{Exemplo de figura com dados importantes}
%   \caption*{Fonte: \cite{silva2020}}
%   \label{fig:exemplo}
% \end{figure}
% 
% No texto: "conforme mostra a Figura \ref{fig:exemplo}..."

% TABELA PROFISSIONAL (se usar booktabs):
% \begin{table}[h]
% \centering
% \caption{Resultados do experimento}
% \label{tab:resultados}
% \begin{tabular}{lrrr}
% \toprule
% Método & Precisão & Recall & F1-Score \\
% \midrule
% Método A & 0.85 & 0.80 & 0.82 \\
% Método B & 0.90 & 0.75 & 0.82 \\
% Método C & 0.88 & 0.85 & 0.86 \\
% \bottomrule
% \end{tabular}
% \caption*{Fonte: Elaboração própria.}
% \end{table}

% LISTA COM SUBITENS:
% \begin{enumerate}
%     \item Primeiro item principal
%     \begin{enumerate}
%         \item Subitem A
%         \item Subitem B
%     \end{enumerate}
%     \item Segundo item principal
%     \item Terceiro item principal
% \end{enumerate}

% LISTA NÃO NUMERADA:
% \begin{itemize}
%     \item Primeiro ponto
%     \item Segundo ponto
%     \item Terceiro ponto
% \end{itemize}

% NOTA DE RODAPÉ:
% Este texto tem uma nota\footnote{Esta é uma nota de rodapé explicativa.}.

% REFERÊNCIA CRUZADA:
% Como visto no Capítulo \ref{chap:metodologia}, Seção \ref{sec:materiais}...

% ===================================================================
% CONFIGURAÇÕES OPCIONAIS ADICIONAIS
% ===================================================================

% QUEBRA DE PÁGINA FORÇADA:
% \newpage

% QUEBRA DE COLUNA (se usar multicol):
% \columnbreak

% ESPAÇAMENTO VERTICAL PERSONALIZADO:
% \vspace{2cm}

% LINHA HORIZONTAL:
% \hrule

% TEXTO EM DESTAQUE:
% \textbf{texto em negrito}
% \textit{texto em itálico}
% \underline{texto sublinhado}
% \textsc{Texto em Versalete}

% MARCADORES ESPECIAIS:
% \checkmark (se amssymb estiver carregado)
% \times para indicar "não"
% \bullet para marcadores simples

% --- FIM DO DOCUMENTO ---
\end{document} 